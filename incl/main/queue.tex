\chapter{Køsimulering} \label{chap:queue}
Når flyene ankommer kan der opstå et tilfælde hvor et fly ankommer mens alle landingabaner er optagede. I det tilfælde skal flyet der ikke endnu kan lande forblive i luften. Dette er et eksempel på et køproblem. Den simpleste måde at løse problemet er altid at lade den forreste i køen lande, når en landingsbane bliver ledig.

Der er mange tilgange til at simulere en kø. I projektet blev der brugt (KILDE) som inspiration til en køsimulering med både én og flere landingsbaner.
I Figur \ref{fig:queue_flowchart} på næste side ses et flowchart som illustrerer den køsimulation der bliver brug i dette projekt.

\begin{figure}
	\centering
	\scalebox{0.7}{
		\tikzstyle{startstop} = [rectangle, rounded corners, minimum width=3cm, minimum height=1cm,text centered, draw=black, fill=red!30]
\tikzstyle{io} = [trapezium, trapezium left angle=70, trapezium right angle=110, minimum width=3cm, minimum height=1cm, text centered, text width=4cm, draw=black, fill=blue!30]
\tikzstyle{process} = [rectangle, minimum width=4cm, minimum height=1cm, text centered, text width=4cm, draw=black, fill=orange!30]
\tikzstyle{decision} = [diamond, width=2cm, height=1cm, text centered, text width=2cm, draw=black, fill=green!30]
\tikzstyle{arrow} = [thick,->,>=stealth]

\begin{tikzpicture}[node distance=2cm]
	\node (start) [startstop] {Start};
	\node (pro1) [process, below of=start] {Sæt tiden og ventetiden lig nul, samt lad alle landingsbaner være tomme};
	\node (dec1) [decision, below of=pro1, yshift=-1cm] {Er alle fly landet?};
	\node (pro2) [process, below of=dec1, yshift=-1cm] {Beregn den næste ankomsttid};
	\node (dec2) [decision, below of=pro2, yshift=-1.5cm] {Er det næste event en ankomst?};
	\node (pro3a) [process, right of=dec2, xshift=3cm] {Sæt tiden lig den næste ankomsttid};
	\node (dec3a) [decision, right of=pro3a, xshift=3cm] {Er der en fri landingsbane?};
	\node (pro4aa) [process, right of=dec3a, xshift=3cm] {Land flyet her og udregn den næste tid en landing er færdig for denne bane};
	\node (pro4ab) [process, below of=dec3a, yshift=-1cm] {Placér flyet bagerst i køen};
	\node (dec4) [decision, below of=dec2, yshift=-5cm] {Er det næste event at en landingsbane bliver fri?};
	\node (pro5a) [process, right of=dec4, xshift=3cm] {Sæt tiden lig den næste tid til en landing er færdig};
	\node (dec5a) [decision, right of=pro5a, xshift=3cm] {Er der fly i køen?};
	\node (pro6aa) [process, right of=dec5a, xshift=3cm] {Tag det forreste fly ud af køen og land flyet her. Udregn den næste tid en landing er flrdig for denne bane, og opdater den samlede ventetid med ventetiden for dette fly};
	\node (pro6ab) [process, below of=dec5a, yshift=-1cm] {Sæt landingsbanen til at være tom};
	\node (out1) [io, below of=dec4, yshift=-4cm] {Udregn den gennemsnittelige ventetid og retuner denne};
	\node (stop) [startstop, below of=out1] {Stop};

	\node (dec4an) [above of=dec4, yshift=1cm];
	\node (dec1ana) [below of=dec4, yshift=-2cm, xshift=-4cm];
	\node (out1an) [above of=out1, xshift=18cm, yshift=-0.7cm];
	\node (nej1) [below of=dec4, anchor=east, yshift=-1cm] {Nej};
	\node (ja1) [right of=dec1, anchor=south] {Ja};
	
	\draw [arrow] (start) -- (pro1);
	\draw [arrow] (pro1) -- (dec1);
	\draw [arrow] (dec1) -- node[anchor=east] {Nej} (pro2);
	\draw [arrow] (pro2) -- (dec2);
	\draw [arrow] (dec2) -- node[anchor=south] {Ja} (pro3a);
	\draw [arrow] (pro3a) -- (dec3a);
	\draw [arrow] (dec3a) -- node[anchor=south] {Ja} (pro4aa);
	\draw [arrow] (dec3a) -- node[anchor=east] {Nej} (pro4ab);
	\draw [thick] (pro4aa) |- (dec4an);
	\draw [thick] (pro4ab) |- (dec4an);
	\draw [arrow] (dec2) -- node[anchor=east] {Nej} (dec4);
	\draw [arrow] (dec4) -- node[anchor=south] {Ja} (pro5a);
	\draw [arrow] (pro5a) -- (dec5a);
	\draw [arrow] (dec5a) -- node[anchor=south] {Ja} (pro6aa);
	\draw [arrow] (dec5a) -- node[anchor=east] {Nej} (pro6ab);
	\draw [thick] (pro6aa) |- (dec1ana);
	\draw [thick] (pro6ab) |- (dec1ana);
	\draw [arrow] (dec1ana) |- (dec1);
	\draw [thick] (dec4) |- (dec1ana);
	\draw [thick] (dec1) -| (out1an);
	\draw [arrow] (out1an) -| (out1);
	\draw [arrow] (out1) -- (stop);
\end{tikzpicture}

	}
	\caption{Flowchat der illustrer en køsimulering} \label{fig:queue_flowchart}
\end{figure}

I starten af simuleringen sættes en variabel hvor den samlede tid der er gået lig nul.
Desuden sættes også den samlede ventetid lig nul, og alle landingsbanerne sættes til at være tomme. Derefter startes en while-løkke, som først sluttet når alle fly har landet.
I while-løkken tjekkes der for to centrale begivenheder (events).
Den ene begivenhed er at et nyt fly ankommer til lufthavnen, mens den anden begivenhed er at en landingsbane bliver ledig.
Hvis et nyt fly ankommer, skal det enten lande på en ledig landingsbane, eller også placeres bagerst i køen.
Hvis en landingsbane bliver ledig, skal det næste fly i køen lande, hvis ikke køen er tom, eller også skal landingsbanen opdateres som tom, hvis der ingen fly er i køen.
Hvis et fly har været i køen, skal den tid det har ventet i køen gemmes når det får lov til at lande. Denne ventetid bliver plusset til den samlede ventetid, og til sidst i simuleringen vil den gennemsnittelige ventetid per fly blive returneret. 
Hvis et et fly ikke har været i kø, har det ikke haft nogen ventetid, og da er det ikke nødvendigt at opdatere den samlede ventetid.

Denne simulation virker for ethvert antal af landingsbaner.
Derfor er det den samme der vil blive brugt både til tilfældet hvor der er én, og hvor der er to landingsbaner.

Køsimuleringen er en del af det større program der udregner den gennemsnittelige ventetid per fly på en dag hvert år.
Dette program kan findes som filen \code{airplanes.py}, og køsimuleringen er en funktion der heri starter på linje 45 kaldet \code{sim\_day()}.
