\chapter{Programmer}
I dette appendiks er alle prgrammer der er blevet nævt i rapporten indsat som kildekode.
Deres funktionanlitet er beskrevet kort i starten af hver afsnit.

\section{arrival\_times.py}
Dette program udregner summen af alle ankomsttider mellem flyene for en tilfædig dag det første år, ved at gøre brug af de givne data.

\begin{lstlisting}[language=Python, caption={arrival\_times.py}]
import random
# The given data
inter_arrival_time_num = [44, 34, 27, 22, 16, 13, 10, 8, 6, 5, 4, 3, 2, 2, 1, 1, 1, 0, 1, 0]

# For each time interval, generate the inter-arrival time number amount of inter-arrival times in this interval
inter_arrival_time = []
for index, time in enumerate(range(0, 1200, 60)):
    for i in range(inter_arrival_time_num[index]):
        inter_arrival_time.append(random.randint(time, time + 59))
# Print results - the number of airplanes (which is the length of inter_arrival_times, and the sum of inter-arrival times in hours
print(f"Numnber of airplanes: {len(inter_arrival_time)}")
print(f"Hours in total: {round(sum(inter_arrival_time) / 3600.0, 2)}")
\end{lstlisting}

\section{arrival\_times\_mean.py}
Dette program gentager funktionaliteten fra \code{arrival\_times.py} mange gange, og printer gennemsnittet.

\begin{lstlisting}[language=Python, caption={arrival\_times\_mean.py}]
import random
repetitions = 1000

# From arrival_times.py:
def arrival_times():
    inter_arrival_time_num = [44, 34, 27, 22, 16, 13, 10, 8, 6, 5, 4, 3, 2, 2, 1, 1, 1, 0, 1, 0]
    inter_arrival_time = []
    for index, time in enumerate(range(0, 1199, 60)):
        for i in range(inter_arrival_time_num[index]):
            inter_arrival_time.append(random.randint(time, time + 59))
    return sum(inter_arrival_time)

# Calculate the mean after repeating the program many times
mean = 0
for r in range(repetitions):
    mean += arrival_times()
mean /= float(repetitions)
# Print the mean
print(f"Average hours in total: {round(mean / 3600.0, 3)}")
\end{lstlisting}

\section{plot\_inter\_arrival\_time\_model.py}
Dette program gentager mange gange at bruge modellen at generere ankomsttider, udregne "inter-arrival times" og optælle hyppigheden i de givne intervaller, og tager da gennemsnittet af hyppighederne.
Desse gennemsnit bliver til slut plottet.

\begin{lstlisting}[language=Python, caption={plot\_inter\_arrival\_time\_model.py}]
import random
import numpy as np
import matplotlib.pyplot as plt

# Generate a list of the number of inter-arrival times
def generate_inter_arrival_times_num(planes):
    # Genreate arrival times using the model
    mylist = [random.randint(0, 13 * 3600) for i in range(planes)]
    mylist.sort()
    # Generate the inter-arrival times
    inter_times = [mylist[0]]
    for index in range(len(mylist[1:])):
        inter_times.append(mylist[index + 1] - mylist[index])
    # Calculate the number of inter-arrival times in the specific intervals
    inter_arrival_times_num = []
    for t in range(0, 1200, 60):
        number = 0
        for time in inter_times:
            if t <= time and time < t + 59:
                number += 1
        inter_arrival_times_num.append(number)
    # Return the list of the number of inter-arrival times
    return inter_arrival_times_num

# Take the mean of a list generated with a function (with one argument) when run many times
def mean_list(repeat, function, arg):
    mean = [0 for i in range(len(function(arg)))]
    for i in range(repeat):
        mean = map(lambda x, y: x + y, mean, function(arg)) 
    return [x/float(repeat) for x in mean]

# Generate the mean of the number of inter-arrival times
repetitions = 20000
planes = 200
performance = mean_list(repetitions, generate_inter_arrival_times_num, planes)
print(performance)

# Plot the data in a bar graph
labels = [str(start) + "-" + str(start + 59) for start in range(0, 1200, 60)]
objects = tuple(labels)
y_pos = np.arange(len(objects))
plt.bar(y_pos, performance, align='center', alpha=1)
plt.xticks(y_pos, objects, rotation=90)
plt.ylabel('Number of planes')
plt.xlabel('Inter-arrival time [seconds]')
plt.title(f'Number of inter-arrival times in the given intervals')
plt.tight_layout()
plt.savefig(f'mean_number_of_inter_arrival_times_{repetitions}_{planes}.png')
\end{lstlisting}

\section{airplanes.py}
Dette program er det centrale program i projektet.
Funktionaliteten gennemsås i Kapitel \ref{chap:program}.
Programmet kører mange køsimuleringer og udregner den genenmsnittelige ventetid per fly.

\begin{lstlisting}[language=Python, caption={airplanes.py}]
import random
import numpy as np
import matplotlib.pyplot as plt

# Genreate a list of landing durations
def generate_landing_durations(planes):
    data = [0, 16, 33, 61, 41, 25, 10, 8, 6]
    min_ = 0
    max_ = 30
    model = [[0.0, 0, 30]]
    for index, num in enumerate(data[1:]):
        percent = float(num) / planes + model[index][0]
        min_ = 1 + max_
        max_ = 29 + min_
        model.append([percent, min_, max_])
    percent_list = [random.random() for i in range(planes)]
    landing_durations = []
    for percent in percent_list:
        for index in range(len(model) - 1):
            if model[index][0] < percent and percent <= model[index + 1][0]:
                landing_durations.append(random.randint(model[index + 1][1], model[index + 1][2]))
                break
    return landing_durations

# Generate a list of arrival times
def generate_arrival_times(planes):
    arrival_times = [random.randint(0, 13 * 3600) for i in range(planes)]
    arrival_times.sort()
    return arrival_times

# Generate schedule that contains a list of lists.
def generate_schedule(planes):
    arrival_times = generate_arrival_times(planes)
    landing_durations = generate_landing_durations(planes)
    return list(zip(arrival_times, landing_durations))

def next_lane_event(lanes):
    next_lane = max(lanes)
    for lane in lanes:
        if lane < next_lane and lane != -1:
            next_lane = lane
    return next_lane

# Run simulation for one day
def sim_day(runways, schedule):
    t = 0
    cur_flight = 0
    lanes = [-1 for runway in range(runways)]
    queue = []
    wait = 0.0
    while(cur_flight < len(schedule) or queue or max(lanes) > -1):
        if cur_flight >= len(schedule):
            next_arrival = 100000000
        else:
            next_arrival = schedule[cur_flight][0]
        if next_arrival <= next_lane_event(lanes) or (next_lane_event(lanes) == -1 and cur_flight < len(schedule)):
            t = next_arrival
            if -1 != min(lanes):
                queue.append(schedule[cur_flight])
            else:
                for index, lane in enumerate(lanes):
                    if lane == -1:
                        lanes[index] = t + schedule[cur_flight][1]
                        break
            cur_flight += 1
        if next_lane_event(lanes) > -1:
            for index, lane in enumerate(lanes):
                if lane < next_arrival and lane == next_lane_event(lanes):
                    t = lane
                    if queue:
                        wait += t - queue[0][0]
                        lanes[index] = t + queue[0][1]
                        queue.pop(0)
                    else:
                        lanes[index] = -1
    return wait/len(schedule)

# Run one day's simulationen sim_per_year times
def sim_year(runways, planes, sim_per_year):
    sum_of_avg_wait = 0
    for sim in range(sim_per_year):
        schedule = generate_schedule(planes)
        sum_of_avg_wait += sim_day(runways, schedule)
    return sum_of_avg_wait/float(sim_per_year)

# Run the simulation and calculate number of planes each day in the year
def simulate(runways, years, sim_per_year):
    planes = 200
    avg_wait = []
    for i in range(years):
        avg_wait.append(sim_year(runways, round(planes), sim_per_year))
        planes *= 1.05
    return avg_wait

# Write to file
def write_results_to_file(avg_wait, years):
    avg_wait = [round(avg, 2) for avg in avg_wait]
    results = ", ".join(map(str, avg_wait))
    years_str = ", ".join(map(str, range(1, years + 1)))
    f = open(f"{years}Y{runways}K.txt", "w")
    f.write(years_str + "\n" + results + "\n")
    f.close()
    print(f"Writing results to file \"{years}Y{runways}K.txt\"")

# Run the simulation and write the results to a file
if __name__ == "__main__":
    years = 15
    runways = 1
    avg_wait = simulate(runways, years, 300)
    write_results_to_file(avg_wait, years)
\end{lstlisting}

\section{plot.py}
Dette program tager et antal filer (genereret med \code{airplanes.py}) som input i kommandolinjen, og plotter data fra filerne i et histogram.
Histogrammet gemmes som en \textsc{PNG}-fil.

\begin{lstlisting}[language=Python, caption={plot.py}]
import argparse
import matplotlib
import matplotlib.pyplot as plt
import numpy as np
import os
import sys

def close_files():
    for f in args.file:
        f.close()

# Turn a line in a file into a list of the type of data the input function returns
def read_line(f, line_num, function):
    f.seek(0)
    lines = f.readlines()
    try:
        mylist = list(map(function, lines[line_num].strip().split(', ')))
    except ValueError:
        print("Error: Make sure the files were generated using airplanes.py.")
        close_files()
        sys.exit()
    return mylist

# Get data from files
def get_data_from_files():
    data = [[] for i in range(len(args.file))]
    num = [] # number of runways
    for index, f in enumerate(args.file):
        # Turn line into a list
        data[index].append(read_line(f, 0, str))
        data[index].append(read_line(f, 1, float))
        # Get the number of runways as str from the filename
        name = os.path.basename(f.name)
        i = name.find("K")
        num.append(name[i - 1])
    close_files()
    return data, num

# Get the labels for the x-axis and check if the files match
def get_labels(data):
    labels = data[0][0]
    for d in data[1:]:
        if d[0] != labels:
            print("Error: The files' labels are not compatible. Make sure the number of years in each file is the same.")
            sys.exit()
    return labels

# Plot the data
def plot(data, labels, num_of_runways):
    x = np.arange(len(labels))
    width = 0.35
    
    fig, ax = plt.subplots()
    rects = []
    for index, d in enumerate(data):
        rects.append(ax.bar((x - width/2) + index*width/len(data), d[1], width, label=(num_of_runways[index] + " runways")))
    
    ax.set_ylabel('Average waiting time [s]')
    ax.set_title('Average waiting time')
    ax.set_xticks(x)
    ax.set_xticklabels(labels)
    ax.set_xlabel('Years')
    ax.legend()
    fig.tight_layout()
    plt.savefig(f"plot_{data[0][0][-1]}Y_{str(len(data))}files.png")
    print(f"Saving plot as file \"plot_{data[0][0][-1]}Y_{str(len(data))}files.png\"")

# Main function
if __name__ == "__main__":
    parser = argparse.ArgumentParser(description='Plot data from files in a bar graph')
    parser.add_argument('file', type=argparse.FileType('r'), nargs='+')
    args = parser.parse_args()

    data, num_of_runways = get_data_from_files()
    labels = get_labels(data)
    plot(data, labels, num_of_runways)
\end{lstlisting}
